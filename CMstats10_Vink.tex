\PassOptionsToPackage{unicode=true}{hyperref} % options for packages loaded elsewhere
\PassOptionsToPackage{hyphens}{url}
%
\documentclass[ignorenonframetext,]{beamer}
\usepackage{pgfpages}
\setbeamertemplate{caption}[numbered]
\setbeamertemplate{caption label separator}{: }
\setbeamercolor{caption name}{fg=normal text.fg}
\beamertemplatenavigationsymbolsempty
% Prevent slide breaks in the middle of a paragraph:
\widowpenalties 1 10000
\raggedbottom
\setbeamertemplate{part page}{
\centering
\begin{beamercolorbox}[sep=16pt,center]{part title}
  \usebeamerfont{part title}\insertpart\par
\end{beamercolorbox}
}
\setbeamertemplate{section page}{
\centering
\begin{beamercolorbox}[sep=12pt,center]{part title}
  \usebeamerfont{section title}\insertsection\par
\end{beamercolorbox}
}
\setbeamertemplate{subsection page}{
\centering
\begin{beamercolorbox}[sep=8pt,center]{part title}
  \usebeamerfont{subsection title}\insertsubsection\par
\end{beamercolorbox}
}
\AtBeginPart{
  \frame{\partpage}
}
\AtBeginSection{
  \ifbibliography
  \else
    \frame{\sectionpage}
  \fi
}
\AtBeginSubsection{
  \frame{\subsectionpage}
}
\usepackage{lmodern}
\usepackage{amssymb,amsmath}
\usepackage{ifxetex,ifluatex}
\usepackage{fixltx2e} % provides \textsubscript
\ifnum 0\ifxetex 1\fi\ifluatex 1\fi=0 % if pdftex
  \usepackage[T1]{fontenc}
  \usepackage[utf8]{inputenc}
  \usepackage{textcomp} % provides euro and other symbols
\else % if luatex or xelatex
  \usepackage{unicode-math}
  \defaultfontfeatures{Ligatures=TeX,Scale=MatchLowercase}
\fi
% use upquote if available, for straight quotes in verbatim environments
\IfFileExists{upquote.sty}{\usepackage{upquote}}{}
% use microtype if available
\IfFileExists{microtype.sty}{%
\usepackage[]{microtype}
\UseMicrotypeSet[protrusion]{basicmath} % disable protrusion for tt fonts
}{}
\IfFileExists{parskip.sty}{%
\usepackage{parskip}
}{% else
\setlength{\parindent}{0pt}
\setlength{\parskip}{6pt plus 2pt minus 1pt}
}
\usepackage{hyperref}
\hypersetup{
            pdftitle={Hybrid imputation},
            pdfauthor={Gerko Vink and Stef van Buuren},
            pdfborder={0 0 0},
            breaklinks=true}
\urlstyle{same}  % don't use monospace font for urls
\newif\ifbibliography
\usepackage{color}
\usepackage{fancyvrb}
\newcommand{\VerbBar}{|}
\newcommand{\VERB}{\Verb[commandchars=\\\{\}]}
\DefineVerbatimEnvironment{Highlighting}{Verbatim}{commandchars=\\\{\}}
% Add ',fontsize=\small' for more characters per line
\usepackage{framed}
\definecolor{shadecolor}{RGB}{248,248,248}
\newenvironment{Shaded}{\begin{snugshade}}{\end{snugshade}}
\newcommand{\AlertTok}[1]{\textcolor[rgb]{0.94,0.16,0.16}{#1}}
\newcommand{\AnnotationTok}[1]{\textcolor[rgb]{0.56,0.35,0.01}{\textbf{\textit{#1}}}}
\newcommand{\AttributeTok}[1]{\textcolor[rgb]{0.77,0.63,0.00}{#1}}
\newcommand{\BaseNTok}[1]{\textcolor[rgb]{0.00,0.00,0.81}{#1}}
\newcommand{\BuiltInTok}[1]{#1}
\newcommand{\CharTok}[1]{\textcolor[rgb]{0.31,0.60,0.02}{#1}}
\newcommand{\CommentTok}[1]{\textcolor[rgb]{0.56,0.35,0.01}{\textit{#1}}}
\newcommand{\CommentVarTok}[1]{\textcolor[rgb]{0.56,0.35,0.01}{\textbf{\textit{#1}}}}
\newcommand{\ConstantTok}[1]{\textcolor[rgb]{0.00,0.00,0.00}{#1}}
\newcommand{\ControlFlowTok}[1]{\textcolor[rgb]{0.13,0.29,0.53}{\textbf{#1}}}
\newcommand{\DataTypeTok}[1]{\textcolor[rgb]{0.13,0.29,0.53}{#1}}
\newcommand{\DecValTok}[1]{\textcolor[rgb]{0.00,0.00,0.81}{#1}}
\newcommand{\DocumentationTok}[1]{\textcolor[rgb]{0.56,0.35,0.01}{\textbf{\textit{#1}}}}
\newcommand{\ErrorTok}[1]{\textcolor[rgb]{0.64,0.00,0.00}{\textbf{#1}}}
\newcommand{\ExtensionTok}[1]{#1}
\newcommand{\FloatTok}[1]{\textcolor[rgb]{0.00,0.00,0.81}{#1}}
\newcommand{\FunctionTok}[1]{\textcolor[rgb]{0.00,0.00,0.00}{#1}}
\newcommand{\ImportTok}[1]{#1}
\newcommand{\InformationTok}[1]{\textcolor[rgb]{0.56,0.35,0.01}{\textbf{\textit{#1}}}}
\newcommand{\KeywordTok}[1]{\textcolor[rgb]{0.13,0.29,0.53}{\textbf{#1}}}
\newcommand{\NormalTok}[1]{#1}
\newcommand{\OperatorTok}[1]{\textcolor[rgb]{0.81,0.36,0.00}{\textbf{#1}}}
\newcommand{\OtherTok}[1]{\textcolor[rgb]{0.56,0.35,0.01}{#1}}
\newcommand{\PreprocessorTok}[1]{\textcolor[rgb]{0.56,0.35,0.01}{\textit{#1}}}
\newcommand{\RegionMarkerTok}[1]{#1}
\newcommand{\SpecialCharTok}[1]{\textcolor[rgb]{0.00,0.00,0.00}{#1}}
\newcommand{\SpecialStringTok}[1]{\textcolor[rgb]{0.31,0.60,0.02}{#1}}
\newcommand{\StringTok}[1]{\textcolor[rgb]{0.31,0.60,0.02}{#1}}
\newcommand{\VariableTok}[1]{\textcolor[rgb]{0.00,0.00,0.00}{#1}}
\newcommand{\VerbatimStringTok}[1]{\textcolor[rgb]{0.31,0.60,0.02}{#1}}
\newcommand{\WarningTok}[1]{\textcolor[rgb]{0.56,0.35,0.01}{\textbf{\textit{#1}}}}
\usepackage{longtable,booktabs}
\usepackage{caption}
% These lines are needed to make table captions work with longtable:
\makeatletter
\def\fnum@table{\tablename~\thetable}
\makeatother
\setlength{\emergencystretch}{3em}  % prevent overfull lines
\providecommand{\tightlist}{%
  \setlength{\itemsep}{0pt}\setlength{\parskip}{0pt}}
\setcounter{secnumdepth}{0}

% set default figure placement to htbp
\makeatletter
\def\fps@figure{htbp}
\makeatother


\title{Hybrid imputation}
\author{Gerko Vink and Stef van Buuren}
\date{Recent advancements in iterative imputation}

\begin{document}
\frame{\titlepage}

\begin{frame}

\end{frame}

\begin{frame}{This presentation \textbar{} has a website:}
\protect\hypertarget{this-presentation-has-a-website}{}

\href{https://www.gerkovink.com/London2019/}{www.gerkovink.com/London2019/}

 You can find all related materials, links and references there.

\end{frame}

\hypertarget{a-short-overview}{%
\section{A short overview}\label{a-short-overview}}

\begin{frame}{Imputation}
\protect\hypertarget{imputation}{}

For those of you who are unfamiliar with imputation:

 With imputation, some estimation procedure is used to impute (fill in)
each missing datum, resulting in a completed dataset that can be
analyzed as if the data were completely observed.

We can do this once (single imputation) or multiple times (multiple
imputation).

\begin{itemize}
\tightlist
\item
  \textbf{With MI, each missing datum is imputed \(m \geq 2\) times,
  resulting in \(m\) completed datasets.}
\end{itemize}

Multiple imputation has some benefits over single imputation:

\begin{itemize}
\tightlist
\item
  it accounts for missing data uncertainty
\item
  it accounts for parameter uncertainty
\item
  additional adjustments to obtain valid inference are not necessary
\end{itemize}

\end{frame}

\begin{frame}{How to go about this?}
\protect\hypertarget{how-to-go-about-this}{}

Once we start the process of multiple imputation, we need a scheme to
solve for multivariate missingness

Some notation:

\begin{itemize}
\item
  Let \(Y\) be an incomplete column in the data

  \begin{itemize}
  \tightlist
  \item
    \(Y_\mathrm{mis}\) denoting the unobserved part
  \item
    \(Y_\mathrm{obs}\) denotes the observed part
  \end{itemize}
\item
  Let \(X\) be a set of completely observed covariates
\end{itemize}

In general, there are two flavours of multiple imputation:

\begin{enumerate}
\tightlist
\item
  We can either model the joint distribution of the data by means of
  \textbf{joint modeling}
\item
  Or, we can model each variable separately by means of \textbf{fully
  conditional specification}
\end{enumerate}

\end{frame}

\begin{frame}{Joint modeling}
\protect\hypertarget{joint-modeling}{}

With JM, imputations are drawn from an assumed joint multivariate
distribution.

\begin{itemize}
\tightlist
\item
  Often a multivariate normal model is used for both continuous and
  categorical data,
\item
  Other joint models have been proposed (see e.g.~Olkin and Tate, 1961;
  Van Buuren and van Rijckevorsel, 1992; Schafer, 1997; Van Ginkel et
  al., 2007; Goldstein et al., 2009; Chen et al., 2011).
\end{itemize}

Joint modeling imputations generated under the normal model are usually
robust to misspecification of the imputation model (Schafer, 1997;
Demirtas et al., 2008), \textbf{although transformation towards
normality is generally beneficial.}

\begin{block}{Procedure}

\begin{enumerate}
\tightlist
\item
  Specify the joint model \(P(Y,X)\)
\item
  Derive \(P(Y_\mathrm{mis}|Y_\mathrm{obs},X)\)
\item
  Draw imputations \(\dot Y^\mathrm{mis}\) with a Gibbs sampler
\end{enumerate}

\end{block}

\end{frame}

\begin{frame}{Joint modeling}
\protect\hypertarget{joint-modeling-1}{}

\textbf{PRO}

\begin{itemize}
\tightlist
\item
  The conditionals are compatible
\item
  The statistical inference is correct under the assumed joint model
\item
  Efficient parametrization is possible
\item
  The theoretical properties are known
\end{itemize}

\textbf{CON}

\begin{itemize}
\tightlist
\item
  Having to specify a joint model impacts flexibility
\item
  The JM can assume more than the complete data problem
\item
  It can lead to unrealistically large models
\item
  The assumed model may not be very close to the data
\end{itemize}

\end{frame}

\begin{frame}{FCS}
\protect\hypertarget{fcs}{}

Multiple imputation by means of FCS does not start from an explicit
multivariate model.

 With FCS, multivariate missing data is imputed by univariately
specifying an imputation model for each incomplete variable, conditional
on a set of other (possibly incomplete) variables.

\begin{itemize}
\tightlist
\item
  the multivariate distribution for the data is thereby implicitly
  specified through the univariate conditional densities
\item
  imputations are obtained by iterating over the conditionally specified
  imputation models.
\end{itemize}

\begin{block}{Procedure}

\begin{itemize}
\tightlist
\item
  Specify \(P(Y^\mathrm{mis} | Y^\mathrm{obs}, X)\)
\item
  Draw imputations \(\dot Y^\mathrm{mis}\) with Gibbs sampler
\end{itemize}

\end{block}

\end{frame}

\begin{frame}[fragile]{FCS}
\protect\hypertarget{fcs-1}{}

The general idea of using conditionally specified models to deal with
missing data has been discussed and applied by many authors

\begin{itemize}
\tightlist
\item
  see e.g.~Kennickell, 1991; Raghunathan and Siscovick, 1996; Oudshoorn
  et al., 1999; Brand, 1999; Van Buuren et al., 1999; Van Buuren and
  Oudshoorn, 2000; Raghunathan et al., 2001; Faris et al., 2002; Van
  Buuren et al., 2006.
\end{itemize}

Comparisons between JM and FCS have been made that indicate that FCS is
a useful and flexible alternative to JM when the joint distribution of
the data is not easily specified (Van Buuren, 2007) and that similar
results may be expected from both imputation approaches (Lee and Carlin,
2010).

\begin{block}{FCS in \texttt{mice}}

\begin{itemize}
\item
  Specify the imputation models
  \(P(Y_j^\mathrm{mis} | Y_j^\mathrm{obs}, Y_{-j}, X)\)

  \begin{itemize}
  \tightlist
  \item
    where Y\_\{−j\} is the set of incomplete variables except Y\_j
  \end{itemize}
\item
  Fill in starting values for the missing data
\item
  And iterate
\end{itemize}

\end{block}

\end{frame}

\begin{frame}{Why I prefer FCS}
\protect\hypertarget{why-i-prefer-fcs}{}

\textbf{PRO}

\begin{itemize}
\tightlist
\item
  FCS is very flexible
\item
  modeling remains close to the data
\item
  one may use a subset of predictors for each column
\item
  work very well in practice
\item
  straightforward to explain to applied researchers
\end{itemize}

\textbf{CON}

\begin{itemize}
\tightlist
\item
  its theoretical properties are only known in special cases
\item
  potential incompatibility of the collection of conditionals with the
  joint
\item
  no computational shortcuts
\end{itemize}

Bold statement:

\[\text{Merging JM and FCS would be better}\]

\end{frame}

\hypertarget{merge-jm-and-fcs}{%
\section{Merge JM and FCS}\label{merge-jm-and-fcs}}

\begin{frame}[fragile]{Hybrids of JM and FCS}
\protect\hypertarget{hybrids-of-jm-and-fcs}{}

We can combine the flexibility of FCS with the appealing theoretical
properties of JM

In order to do so, we need to partition the variables into
\textbf{blocks}

\begin{itemize}
\item
  For example, we might partition \(b\) blocks \(h = 1,\dots,b\) as
  follows

  \begin{itemize}
  \item
    a single block with \(b=1\) would hold a \textbf{joint model}:
    \[\{Y_1, Y_2, Y_3, Y_4\}, X\]
  \item
    a quadruppel block with \(b=4\) would be the \texttt{mice} algorithm
    \[\{Y_1\},\{Y_2\},\{Y_3\},\{Y_4\}, X\]
  \item
    anything in between would be a hybrid between the joint model and
    the \texttt{mice} model. For example,
    \[\{Y_1, Y_2, Y_3\},\{Y_4\}, X\]
  \end{itemize}
\end{itemize}

\end{frame}

\begin{frame}{Why is this useful}
\protect\hypertarget{why-is-this-useful}{}

It is not rare that sets of variable follow a deterministic relation or
system. Examples are

\begin{itemize}
\tightlist
\item
  \textbf{Imputing squares/nonlinear effects}: In the model
  \(y=\alpha + \beta_1X+\beta_2X^2 + \epsilon\), \(X\) and \(X^2\)
  should be imputed jointly (Von Hippel, 2009, Seaman, Bartlett \&
  White, 2012, Vink \& Van Buuren, 2013, Bartlett et al., 2015)
\item
  \textbf{Compositional data}: Predictive ratio matching (Vink, 2015,
  Ch5)
\end{itemize}

\[
\begin{array}{lllllllllllll}
x_0 &=  &x_1        &+  &x_2        &+      &x_3        &+& x_4 &       &   &   &\\
       &    &=      &       &           &       &       && =        &       &   &   &\\
       &    &x_9        &       &           &       &       && x_5  &       &   &   &\\
       &    &+      &       &           &       &       && +        &       &   &   &\\
       &    &x_{10}     &       &           &       &       &&x_6       &=      &x_7    &+&x_8
\end{array}
\]

\begin{itemize}
\tightlist
\item
  \textbf{Multivariate PMM}: Imputing a combination of outcomes
  optimally based on a linear combination of covariates (Cai, Vink \&
  Van Buuren - working paper).
\end{itemize}

\end{frame}

\begin{frame}{JM embedded within FCS}
\protect\hypertarget{jm-embedded-within-fcs}{}

\begin{longtable}[]{@{}lllll@{}}
\toprule
b & h & target & predictors & type\tabularnewline
\midrule
\endhead
2 & 1 & \(\{Y_1, Y_2, Y_3\}\) & \(Y_4, X\) & mult\tabularnewline
2 & 2 & \(Y_4\) & \(Y_1, Y_2, Y_3, X\) & univ\tabularnewline
\bottomrule
\end{longtable}

\end{frame}

\begin{frame}{FCS embedded within FCS}
\protect\hypertarget{fcs-embedded-within-fcs}{}

\begin{longtable}[]{@{}llllll@{}}
\toprule
b & h & j & target & predictors & type\tabularnewline
\midrule
\endhead
2 & 1 & 1 & \(Y_1\) & \(Y_2, Y_3, Y_4, X\) & univ\tabularnewline
2 & 1 & 2 & \(Y_2\) & \(Y_1, Y_3, Y_4, X\) & univ\tabularnewline
2 & 1 & 3 & \(Y_3\) & \(Y_1, Y_2, Y_4, X\) & univ\tabularnewline
2 & 2 & 1 & \(Y_4\) & \(Y_1, Y_2, Y_3, X\) & univ\tabularnewline
\bottomrule
\end{longtable}

\end{frame}

\begin{frame}{Benefits of blocks in \texttt{mice()}}
\protect\hypertarget{benefits-of-blocks-in-mice}{}

\begin{enumerate}
\tightlist
\item
  Looping over \(b\) blocks instead of looping over \(p\) columns.
\item
  Only specify \(b \times p\) predictor relations and not \(p^2\).
\item
  Only specify \(b\) univariate imputation methods instead of \(p\)
  methods.
\item
  Ability for imputing more than one column at once
\item
  Simplified overall model specification
\end{enumerate}

\begin{itemize}
\tightlist
\item
  e.g.~sets of items in scales, matching items in longitudinal data,
  joining data sets, etc.
\end{itemize}

\end{frame}

\begin{frame}[fragile]{\texttt{predictorMatrix} simplification:}
\protect\hypertarget{predictormatrix-simplification}{}

Under the conventional FCS predictor specification, we could hypothesize
the following \texttt{predictorMatrix}.

\begin{verbatim}
##           age item1 item2 sum_items time1 time2 time3 mean_time
## age         0     0     0         1     0     0     0         1
## item1       1     0     1         0     0     0     0         1
## item2       1     1     0         0     0     0     0         1
## sum_items   0     1     1         0     0     0     0         0
## time1       1     0     0         1     0     1     1         0
## time2       1     0     0         1     1     0     1         0
## time3       1     0     0         1     1     1     0         0
## mean_time   0     0     0         0     1     1     1         0
\end{verbatim}

\end{frame}

\begin{frame}[fragile]{\texttt{predictorMatrix} simplification:}
\protect\hypertarget{predictormatrix-simplification-1}{}

Under the new \texttt{blocked} approach, however, we could simplify
these specifications into the following blocks and predictor relations.

\begin{Shaded}
\begin{Highlighting}[]
\NormalTok{blocks <-}\StringTok{ }\KeywordTok{list}\NormalTok{(}\DataTypeTok{age =} \StringTok{"age"}\NormalTok{, }
               \DataTypeTok{A =} \KeywordTok{c}\NormalTok{(}\StringTok{"item1"}\NormalTok{, }\StringTok{"item2"}\NormalTok{, }\StringTok{"sum_items"}\NormalTok{), }
               \DataTypeTok{B =} \KeywordTok{c}\NormalTok{(}\StringTok{"time1"}\NormalTok{, }\StringTok{"time2"}\NormalTok{, }\StringTok{"time3"}\NormalTok{, }\StringTok{"mean_time"}\NormalTok{))}
\end{Highlighting}
\end{Shaded}

\begin{verbatim}
##       age item1 item2 sum_items time1 time2 time3 mean_time
## age     0     0     0         1     0     0     0         1
## Items   1     0     0         0     0     0     0         1
## Time    1     0     0         1     0     0     0         0
\end{verbatim}

\end{frame}

\begin{frame}{Multilevel imputation}
\protect\hypertarget{multilevel-imputation}{}

\end{frame}

\hypertarget{overview}{%
\section{Overview}\label{overview}}

\begin{frame}{Conclusion}
\protect\hypertarget{conclusion}{}

\end{frame}

\end{document}
